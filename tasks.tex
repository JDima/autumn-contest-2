\documentclass[a4]{article}
\pagestyle{myheadings}
\setlength{\parindent}{5ex}
%%%%%%%%%%%%%%%%%%%
% Packages/Macros %
%%%%%%%%%%%%%%%%%%%
\usepackage{mathrsfs}


\usepackage{fancyhdr}
\pagestyle{fancy}
\lhead{}
\chead{}
\rhead{}
\lfoot{}
\cfoot{} 
\rfoot{\normalsize\thepage}
\renewcommand{\headrulewidth}{0pt}
\renewcommand{\footrulewidth}{0pt}
\newcommand{\RomanNumeralCaps}[1]
    {\MakeUppercase{\romannumeral #1}}

\usepackage{amssymb,latexsym}  % Standard packages
\usepackage[utf8]{inputenc}
\usepackage[russian]{babel}
\usepackage{MnSymbol}
\usepackage{mathrsfs}
\usepackage{amsmath,amsthm}
\usepackage{indentfirst}
\usepackage{graphicx}%,vmargin}
\usepackage{graphicx}
\graphicspath{{pictures/}} 
\usepackage{verbatim}
\usepackage{color}
\usepackage{color,colortbl}
\usepackage[nottoc,numbib]{tocbibind}
\usepackage{float}
\usepackage{multirow}
\usepackage{hhline}

\usepackage{listings}
\definecolor{codegreen}{rgb}{0,0.6,0}
\definecolor{codegray}{rgb}{1,1,1}
\definecolor{codepurple}{rgb}{0.58,0,0.82}
\definecolor{backcolour}{rgb}{0.95,0.95,0.92}
 
\lstdefinestyle{mystyle}{
    backgroundcolor=\color{backcolour},   
    commentstyle=\color{codegreen},
    keywordstyle=\color{magenta},
    numberstyle=\tiny\color{codegray},
    stringstyle=\color{codepurple},
    basicstyle=\footnotesize,
    breakatwhitespace=false,         
    breaklines=true,                 
    captionpos=b,                    
    keepspaces=true,                 
    numbers=left,                    
    numbersep=5pt,                  
    showspaces=false,                
    showstringspaces=false,
    showtabs=false,                  
    tabsize=2
}
 
\lstset{style=mystyle}

\usepackage{url}
\urldef\myurl\url{foo%.com}
\def\UrlBreaks{\do\/\do-}
\usepackage{breakurl}
\Urlmuskip=0mu plus 1mu



\DeclareGraphicsExtensions{.pdf,.png,.jpg}% -- настройка картинок

\usepackage{epigraph} %%% to make inspirational quotes.
\usepackage[all]{xy} %for XyPic'a
\usepackage{color} 
\usepackage{amscd} %для коммутативных диграмм
%\usepackage[colorlinks,urlcolor=red]{hyperref}

%\renewcommand{\baselinestretch}{1.5}
%\sloppy
%\usepackage{listings}
%\lstset{numbers=left}
%\setmarginsrb{2cm}{1.5cm}{1cm}{1.5cm}{0pt}{0mm}{0pt}{13mm}


\newtheorem{Lemma}{Лемма}[section]
\newtheorem{Proposition}{Предложение}[section]
\newtheorem{Theorem}{Теорема}[section]
\newtheorem{Corollary}{Следствие}[section]
\newtheorem{Remark}{Замечание}[section]
\newtheorem{Definition}{Определение}[section]
\newtheorem{Designations}{Обозначение}[section]




%%%%%%%%%%%%%%%%%%%%%%% 
%Подготовка оглавления% 
%%%%%%%%%%%%%%%%%%%%%%% 
\usepackage[titles]{tocloft}
\renewcommand{\cftdotsep}{2} %частота точек
\renewcommand\cftsecleader{\cftdotfill{\cftdotsep}}
\renewcommand{\cfttoctitlefont}{\hspace{0.38\textwidth} \LARGE\bfseries} 
\renewcommand{\cftsecaftersnum}{.}
\renewcommand{\cftsubsecaftersnum}{.}
\renewcommand{\cftbeforetoctitleskip}{-1em} 
\renewcommand{\cftaftertoctitle}{\mbox{}\hfill \\ \mbox{}\hfill{\footnotesize Стр.}\vspace{-0.5em}} 
%\renewcommand{\cftchapfont}{\normalsize\bfseries \MakeUppercase{\chaptername} } 
%\renewcommand{\cftsecfont}{\hspace{1pt}} 
\renewcommand{\cftsubsecfont}{\hspace{1pt}} 
%\renewcommand{\cftbeforechapskip}{1em} 
\renewcommand{\cftparskip}{3mm} %определяет величину отступа в оглавлении
\setcounter{tocdepth}{5} 
\renewcommand{\listoffigures}{\begingroup %добавляем номер в список иллюстраций
\tocsection
\tocfile{\listfigurename}{lof}
\endgroup}
\renewcommand{\listoftables}{\begingroup %добавляем номер в список иллюстраций
\tocsection
\tocfile{\listtablename}{lot}
\endgroup}


%\renewcommand{\thelikesection}{(\roman{likesection})}
%%%%%%%%%%%
% Margins %
%%%%%%%%%%%
\addtolength{\textwidth}{0.7in}
\textheight=630pt
\addtolength{\evensidemargin}{-0.4in}
\addtolength{\oddsidemargin}{-0.4in}
\addtolength{\topmargin}{-0.4in}

%%%%%%%%%%%%%%%%%%%%%%%%%%%%%%%%%%%
%%%%%%Переопределение chapter%%%%%% 
%%%%%%%%%%%%%%%%%%%%%%%%%%%%%%%%%%%
\newcommand{\empline}{\mbox{}\newline} 
\newcommand{\likechapterheading}[1]{ 
\begin{center} 
\textbf{\MakeUppercase{#1}} 
\end{center} 
\empline} 

%%%%%%%Запиливание переопределённого chapter в оглавление%%%%%% 
\makeatletter 
\renewcommand{\@dotsep}{2} 
\newcommand{\l@likechapter}[2]{{\bfseries\@dottedtocline{0}{0pt}{0pt}{#1}{#2}}} 
\makeatother 
\newcommand{\likechapter}[1]{ 
\likechapterheading{#1} 
\addcontentsline{toc}{likechapter}{\MakeUppercase{#1}}} 




\usepackage{xcolor}
\usepackage{hyperref}
\definecolor{linkcolor}{HTML}{000000} % цвет ссылок
\definecolor{urlcolor}{HTML}{AA1622} % цвет гиперссылок
 
\hypersetup{pdfstartview=FitH,  linkcolor=linkcolor,urlcolor=urlcolor, colorlinks=true}

%%%%%%%%%%%%
% Document %
%%%%%%%%%%%%

%%%%%%%%%%%%%%%%%%%%%%%%%%%%%
%%%%%%главы -- section*%%%%%%
%%%%section -- subsection%%%%
%subsection -- subsubsection%
%%%%%%%%%%%%%%%%%%%%%%%%%%%%%
\def \newstr {\medskip \par \noindent} 
\begin{document}



\section*{Задача 1}
\label{sec:orgb62fe60}
\subsection*{Постановка}
\label{sec:org37954e9}
Задан неориентированный взвешенный граф, вершины которого пронумерованы от 1 до $n$. Ваша задача найти кратчайший путь из вершины $i$ в вершину $j$.
\subsection*{Входные данные}
\label{sec:orgc51833b}
В первой строке содержатся 2 числа $i$ и  $j$.
В $m$ строках содержатся сами ребра, по одному в строке. Каждое ребро задается тремя числами $a_i$,$b_i$,$w_i$ 
\subsection*{Выходные данные}
\label{sec:org91cd1c2}
Верните кратчайший путь или None, если такого пути нет.
\subsection*{Пример 1}
\label{sec:org1b720b0}

\begin{table}[H]
\begin{center}
\begin{tabular}{|m{4cm}|m{4cm}|}
\hline
Входные данные & Выходные данные \\ \hline
1 4

1 2 1

1 3 4

 1 4 5

 2 3 1

 2 4 4

 3 4 1
&
1 2 3 4
\\ \hline
\end{tabular}
\end{center}
\end{table}


\pagebreak
\section*{Задача 2}
\label{sec:orgef181bd}
\subsection*{Постановка}
\label{sec:orgad8a20e}
У Димы есть полный взвешенный граф. Дима считает сумму длин кратчайших путей между всеми парами вершин. После чего Дима убирает одну из вершин и ребра входящие в эту вершину. Цикл действий повторяется $n$ раз. Помогите Диме посчитать искомую сумму на каждом шаге.
\subsection*{Входные данные}
\label{sec:orgc51833b}
В первой строке содержится целое число $n$ - количество вершин в графе.

В следующих $n$ строках представлена матрица $A$ смежности графа, где $a_{ij}$ - вес ребра.

$1\leq n \leq 500$, $0\leq a_{ij} \leq 10^5$

В последней строке содержится последовательность удаляемых вершин.
\subsection*{Выходные данные}
\label{sec:orgf9da829}
Выведите $n$ чисел каждое из которых - сумма длин кратчайших путей между всеми парами вершин, на $i$-ом шаге. 


\subsection*{Пример 1}
\label{sec:orgd7d348d}

\begin{table}[H]
\begin{center}
\begin{tabular}{|m{4cm}|m{4cm}|}
\hline
Входные данные & Выходные данные \\ \hline
2 

0 5 

4 0

1 2

&
9 0
\\ \hline
\end{tabular}
\end{center}
\end{table}
\pagebreak
\section*{Задача 3}
\label{sec:org570b899}
\subsection*{Постановка}
\label{sec:orga2b5149}
Внуки подарили бабушке на день рождения сотовый телефон. Они записали свои имена и номера телефона в контакты, что бы бабушка всегда видела кто ей звонит и пишет. На следующий день происходит $m$ событий, каждое из которых будет одного из трёх типов:

1. Внук $x$ отправляет бабушке сообщение.

2. Бабушка просматривает все сообщения от внука $x$.

3. Бабушка просматривает первые $t$ сообщений. Бабушка не может определить уже прочитанные поэтому каждый раз смотрит с самого начала.

Помогите бабушке определить количество непросмотренных сообщений после каждого события. Считайте, что изначально никаких сообщений в телефоне не было.

\subsection*{Входные данные}
\label{sec:orgeb4908d}
В первой строке входных данных записаны целые числа $n$ и $m$ - количество внуков и количество событий соответственно. $0\leq n \leq 1000$,$1 \leq m \leq 300000$.

Следующие $m$ строк содержат описания событий, в i-й из них будет сначала записано целое число $type_i$ — тип события. Если $type_i$=1 или $type_i$=2, то далее следует целое число $x_i$. А если $type_i$=3, то далее следует целое число $t_i$ 
\subsection*{Выходные данные}
\label{sec:orged795e8}
Выведите количество непрочитанных сообщений после каждого события.
\subsection*{Пример 1}
\label{sec:org6a26c04}

\begin{table}[H]
\begin{center}
\begin{tabular}{|m{4cm}|m{4cm}|}
\hline
Входные данные & Выходные данные \\ \hline
3 4

1 3

1 1

1 2

2 3
&
1 2 3 2
\\ \hline
\end{tabular}
\end{center}
\end{table}
\pagebreak
\section*{Задача 4}
\label{sec:org570b899}
\subsection*{Постановка}
\label{sec:orga2b5149}
Задана строка, состоящая из букв латинского алфавита и алгоритм действий. Алгоритм состоит из действий двух типов:

1. Заменить i-ый символ строки на другой.

2. Посчитать и вывести количество различных символов в подстроке определенной длины.

Реализуйте алгоритм.
\subsection*{Входные данные}
\label{sec:orgeb4908d}
Первая строка содержит одну строку $s$ длины $n$ ($0\leq n \leq 10^5$).
Далее идет последовательность команд.
\begin{itemize}
    \item 1 pos $c$ - команда s[pos]=c
    \item 2 l r - команда считает и выводи количество различных символов в подстроке s[l:r]
\end{itemize}

\subsection*{Выходные данные}
\label{sec:orged795e8}
Выведите результат работы алгоритма.
\subsection*{Пример 1}
\label{sec:org6a26c04}

\begin{table}[H]
\begin{center}
\begin{tabular}{|m{4cm}|m{4cm}|}
\hline
Входные данные & Выходные \href{https://youtu.be/dQw4w9WgXcQ}{данные} \\ \hline
abacaba

2 1 4

1 4 b

1 5 b

2 4 6

2 1 7
&
3 1 2
\\ \hline
\end{tabular}
\end{center}
\end{table}
\pagebreak

\end{document}